% https://tex.stackexchange.com/questions/291627/draw-an-animated-gif-of-trigonometry-function

\documentclass[tikz]{standalone}

\usepackage{tikz}
\usepackage{pgfplotstable, filecontents}

\begin{document}

% Define and set angle counter for animation;
% here, mean anomaly is used as counter

\newcounter{M}
\setcounter{M}{0}

% Other definitions

\def\semiMajor{1.5}	% Horizontal distance for ellipse
\def\semiMinor{1}		% Vertical distance for ellipse
\def\R{3} 			% Object size (in pt)

% IMPORTANT NOTE: This animation uses the two files ``area.dat'' and
% ``eccentric.dat'', which must be available to the complier.

% I computed this data in MS Excel using Newton's method
% three times (see Wikipedia article on Kepler's equation);
% spreadsheet available as ``Kepler's 2nd law calculations.xlsx''

% Convert inputted data into Tikz-readable format

\pgfplotstableread{eccentric.dat}{\Etable}
\pgfplotstableread{area.dat}{\Eftable}

\begin{center}

\foreach \M in {0, 1, ..., 360}
{
	\begin{tikzpicture}[scale = 0.5]
		
		% Bounding box
		
		\useasboundingbox (-1.2 * \semiMajor, - 1.2 * \semiMinor) rectangle (1.2 * \semiMajor, 1.2 * \semiMinor);
		
		% Find E value of particular frame for planet
		
		\pgfplotstablegetelem{\M}{[index] 0}\of{\Etable}
		\let\E\pgfplotsretval
		
		% Find E value for area swept out by planet
		
		\pgfplotstablegetelem{\M}{[index] 0}\of{\Eftable}
		\let\Ef\pgfplotsretval
		
		% Plot boundaries of area slices beforehand
		
		\foreach \Ebd in {0, 70.2275829, 101.8118571, 124.9878875, 144.6862244, 162.6997767, 180, 197.3002233, 215.3137756, 235.0121125,
			258.1881429, 289.7724171}
			\draw [thin, gray] ({sqrt(\semiMajor * \semiMajor - \semiMinor * \semiMinor)}, 0) -- ({\semiMajor * cos(\Ebd)}, {\semiMinor * sin(\Ebd)});
		
		% Area swept out by planet
		
		\filldraw [green!30] ({sqrt(\semiMajor * \semiMajor - \semiMinor * \semiMinor)}, 0) -- ({\semiMajor * cos(\E)}, {\semiMinor * sin(\E)})
			arc [x radius = \semiMajor, y radius = \semiMinor, end angle = \Ef,
			start angle = \E] -- cycle;
		
		% Elliptical orbit
		
		\draw [thick] (0, 0) ellipse [x radius = \semiMajor, y radius = \semiMinor];
		
		% Sun at right focus
		
		\shade [ball color = yellow] ({sqrt(\semiMajor * \semiMajor - \semiMinor * \semiMinor)}, 0) circle (2 * \R pt);
		
		% Planet
		
		\shade [ball color = blue!50] ({\semiMajor * cos(\E)}, {\semiMinor * sin(\E)}) circle (\R pt);
	
	\end{tikzpicture}
}
\end{center}

\end{document}