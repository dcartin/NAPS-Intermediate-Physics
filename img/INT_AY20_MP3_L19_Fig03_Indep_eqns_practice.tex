\documentclass{standalone}

\usepackage{circuitikz}

\begin{document}

% INT_AY20_MP3_L19_Fig03_Indep_eqns_practice.png

\begin{circuitikz}

	\ctikzset { bipoles/length = 1 cm}
	
	% Actual circuit
	
	\draw (0, 0) node [below left] {$A$} to [R, *-] (0, 3) to [R, -*] (3, 3) node [above] {$B$} to [R] (6, 3) to [battery1] (6, 1.5) to [R, *-] (3, 1.5)
		(0, 0) to [R] (3, 3) -- (3, 1.5) node [above left] {$D$} to [R, *-*] (3, 0) node [below] {$C$}
		(6, 1.5) node [above right] {$E$} -- (6, 0) to [R] (3, 0) to [battery1] (0, 0);
		
	% Circuit arrows and labels
	
	\begin{scope}[->, > = latex, very thick, blue]
	
		\draw [rounded corners] (-0.15, 2.5) -- (-0.15, 3.15) node [above left] {$I_1$} -- (0.5, 3.15);
		\draw (1, 1.5) -- node [midway, above left] {$I_2$} ++ (45 : 1);
		\draw (1, -0.5) -- node [midway, below] {$I_3$} (2, -0.5);
		\draw (4, 3.375) -- node [midway, above] {$I_4$} (5, 3.375);
		\draw (3.15, 1.75) -- node [midway, right] {$I_5$} (3.15, 2.75);
		\draw (2.625, 0.25) -- node [midway, left] {$I_6$} (2.625, 1.25);
		\draw (4, 1.875) -- node [midway, above] {$I_7$} (5, 1.875);
		\draw (6.15, 0.25) -- node [midway, right] {$I_8$} (6.15, 1.25);
	
	\end{scope}

\end{circuitikz}

\end{document}